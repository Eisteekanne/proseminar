\documentclass[11pt]{article}
\usepackage[ngerman]{babel}
\usepackage[utf8]{inputenc}

\begin{document}

\begin{titlepage}
\newcommand{\HRule}{\rule{\linewidth}{0.5mm}} % Defines a new command for the horizontal lines, change thickness here

\center 

\textsc{\LARGE Karlsruher Institut für Technologie}\\[1.5cm] % Name of your university/college
\textsc{\Large Proseminar}\\[0.5cm] % Major heading such as course name
\textsc{\large Informatik in der Medizin}\\[0.5cm] % Minor heading such as course title

\HRule \\[0.4cm]
{ \huge \bfseries Mechanische Eigenschaften von Weichgewebe}\\[0.4cm] % Title of your document
{ \LARGE Gliederung}\\[0.4cm]
\HRule \\[1.5cm]
 

\begin{minipage}{0.4\textwidth}
\begin{flushleft} \large
\emph{Autor:}\\
 Lena Winter 
\end{flushleft}
\end{minipage}
~
\begin{minipage}{0.4\textwidth}
\begin{flushright} \large
\emph{Betreuer:} \\
Jan Hergenhan
\end{flushright}
\end{minipage}\\[4cm]


%----------------------------------------------------------------------------------------
%	DATE SECTION
%----------------------------------------------------------------------------------------

{\large \today}\\[3cm] % Date, change the \today to a set date if you want to be precise

%----------------------------------------------------------------------------------------
%	LOGO SECTION
%----------------------------------------------------------------------------------------

%\includegraphics{Logo}\\[1cm] % Include a department/university logo - this will require the graphicx package
 
%----------------------------------------------------------------------------------------

\vfill % Fill the rest of the page with whitespace

\end{titlepage}
	\section{Motivation:}
		Es soll darauf eingegangen werden, dass das Verhalten von Weichgeweben insbesondere Organen
		sehr komplex ist und es intensiver Untersuchungen bedarf um es zu verstehen.
	\section{Grundlagen:}
			Der Begriff der Viskoelatizität soll erklärt werden und dabei auch die Begriffe der
			Hysterese, der Relaxion und des Kriechens. Dazu soll kurz auf die drei gängigsten 
			viskoeslastischen  Modelle, der MAXWELL-, der VOIGT- und
			der KELVIN-Körper, eingegangen werden.

	\section{Messverfahren: Oszillationsversuch}
		In diesem Abschnitt soll die generele Zielsetzung von Oszillationsversuchen erklärt werden
		und der Aufbau exemplarisch an einem Versuch erklärt werden. Dabei soll auch auf die
		verwendeten Elemente des Aufbaus und ihren Zweck eingeganen werden, wie zum Beispiel 
		künstliche Durchblutung des Versuchsgewebes.
		
	\section{Anwendung}
		In diesem Abschnitt soll die Anwendungen der Messergebnisse erläutert werden. 
		\subsection{Entwicklung haptischer Bedienelemente:}
			Besonderes Augenmerk soll dabei auf die Entwicklung von Bedienelementen für 
			Operationsroboter, die in der minimal invasiven Chirugie eingesetzt werden, die 
			ein haptisches Feedback für den Chirugen liefern.
			
		\subsection{Trainingsprogramme für robotergestütze Chirugie:}
			Es soll darauf eingegangen werden, dass für solche Trainingsprogramme, das Verhalten von
			Organen während eine Operation naturgemäß simuliert werden muss und das für eine solche 
			Simulation die experimentell gewonnen Parameter zur Beschreibung der mechanischen 
			Eigenschaften des Gewebes essentiell sind.
			
	\section{Fazit:}
			Es sollen die Ergebnisse der bisherigen Betrachtungen nocheinmal aufbereitet und 
			zusammengefasst dargelegt werden. Zusätzlich soll ein Ausblick gegeben werden was 
			mit dieser und ähnlicher Forschung in dem Gebiet der minimal invasiven Chirugie erreicht
			werden kann.   
			 
		  
\end{document}
